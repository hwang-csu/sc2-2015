\section{Related Work}
%It is believed that 1.4 billion smartphones will be in active by the
%end of 2013~\cite{rbiresearch}, and 57\% of them will run the Android
%operating system.
The popularity of the Android devices has made it the top target of the mobile
malware~\cite{Felt,Becher,xjiang_oak12} and other potential
attacks~\cite{Fahl,Lu_ccs12}. To improve smartphone security, researchers have
proposed a variety of security frameworks based on application permissions.
Apex~\cite{Nauman} proposes an Android enforcement frame that enables flexible
permission granting and resource restrictions. TISSA~\cite{xjiang_11} protects
the private information leakage by adding an extra permission layer.
Saint~\cite{Ongtang} uses design-time security policies to manage permissions.
Jeon et al.~\cite{jmv+12} propose a set of finer-grained permission for
Android.  All of these frameworks delegate the policy specification to either
the developer or the mobile device user. However, our system delegates the
policy specific to the data owners that send the information to the device.
Other security frameworks attempt to preserve privacy by either modifying data,
or blocking exfiltration.  MockDroid~\cite{Beresford} protects the privacy by
adding perturbations to sensitive data to achieve ambiguity.
AppFence~\cite{Hornyack} provides mechanisms to swap shadow data for private
data, and block data exfiltration.  However, both MockDroid and AppFence only
provide protection against predefined data types on the device (e.g., location,
device id). In contast, DASF allows for the policy to be enforced on any
primitive data type on the device.  ASM~\cite{hnes14} provides an programmable
interface for creating new reference monitors on Android. Our work benefits
from ASM, as we could implement our system over ASM to ensure complete
mediation of our enforcement hooks.

There has been an extensive amount of research focused on tracking and
controlling the flow of private information throughout Android.
FlowDroid~\cite{arf+14} uses static analysis to track information flows in an
application. BayesDroid~\cite{oj14} transforms the problem of data flow
analysis into a statisitical classification problem.
TaintDroid~\cite{taintdroid} extends Android with taint tracking to detect
potential privacy leaks.  We adopt TaintDroid into our framework to enforce the
data access privilege control and propagation restrictions.
CleanOS\cite{tab+13} limits private data disclosure by providing a mechanism
called idle-eviction, which encrypts private data in memory after a timeout
occurs. Our work would benefit from CleanOS's idle eviction mechanism, as we
could selectively evict private data in RAM at a finer granularity.
Aquifer~\cite{ne13} allows applications to set policies on their data to
prevent accidental data disclosures by data intermediaries. The main
distinction between Aquifer and DASF is that our aproach tracks information
flows at a finer-granularity (i.e., variable-level) while Aquifer tracks
information flows at the granularity of a process. 


% To improve smartphone security, a number of
%Android system extensions
%\cite{Nauman,Beresford,xjiang_11,Hornyack,Ongtang,Enck_ccs09,Lange,Andrus}
%have been proposed. Apex~\cite{Nauman} proposes an Android enforcement frame
%that enables flexible permission granting and resource restrictions.
%MockDroid~\cite{Beresford} is closely related to this work in that the Android
%system is modified to protect the privacy.  The main difference, however, is
%that MockDroid protects the privacy by adding perturbations to achieve
%ambiguity.  Our framework offers a system-wide, dynamic and strict access
%restrictions to protect the sensitive data.  In addition, our framework is
%100\% transparent to general Android applications.  TISSA~\cite{xjiang_11}
%protects the private information leakage by an extra permission layer.
%AppFence~\cite{Hornyack} provides a similar fine-grained permission control.
%Different from their schemes, our security framework is more versatile,
%protecting both outgoing and incoming data through on-demand dynamic security
%provisioning.  Saint~\cite{Ongtang} uses design-time security policies to
%manage permissions. Kirin~\cite{Enck_ccs09} proposes a set of security rules to
%mitigate malware.  L4Android~\cite{Lange} and Cells~\cite{Andrus} provide
%improved OS isolation for security. 

%TaintDroid~\cite{taintdroid} provides a information flow monitoring
%system to detect the potential privacy leakage. This idea is adopted
%in our system framework to enforce the data access privilege and
%propogation restrictions.

%TaintDroid~\cite{taintdroid} provides a information flow monitoring system to
%detect the potential privacy leakage. This idea is adopted in our system
%framework to enforce the data access privilege control and propagation
%restrictions.  Static analysis has been used to detect privacy leaks in
%Android applications~\cite{Batyuk} and potential security
%vulnerabilities~\cite{Lu_ccs12}.  Crowdroid~\cite{Burguera} analyzes the system
%calls dynamically to detect malware.  Android permission specification has been
%thoroughly studied in Stowaway~\cite{Felt_ccs11} and PScout~\cite{Au_ccs12}.
%As discussed previously, the permission based Android security system is
%insufficient to address the security challenges in the application compliant
%with the HIPAA rule.

%

