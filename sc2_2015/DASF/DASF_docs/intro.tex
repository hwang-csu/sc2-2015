\section{Introduction}

The incorporation of mobile devices in the healthcare industry
can prove to be another beneficial tool for medical professionals
in a variety ways. For example, medical professionals can use
mobile devices as a general reference tool to assist them with
accurately diagnosing patients, as an informative tool to educate
patients on the treatment of their condition, as a communicative tool
to collaborate with other medical professionals, or to allow
quick access to a patient’s medical records in an emergency.
Furthermore, utilizing mobile devices to allow medical professionals
to access a patient’s medical records also provides the
additional benefit of obviating the need of paper-based medical
records during a patient’s visitation. In regard to the sensitivity
of medical records, electronic medical records are superior
to paper-based medical records because the file server can
record a comprehensive access log of every individual that has
requested the medical record and can also enforce automated
authorization controls to prevent unnecessary or unauthorized
access.

As the healthcare practice is collaborative in nature, doctors
and nurses frequently communicate with each other to provide
a diagnosis and treatment. While the appeal of modern
communication through mobile devices, e.g., smartphones
or tablets, has reached almost every industry, the following
observations witness the status of mobile devices in healthcare
industry. First, a significant number of physicians and nurses
are still relying on traditional clinical communication equipment,
including pagers, phones and fax machines, yet a last
generation wireless communication technology. Second, even
though as many as 81\% healthcare prefessionals own and use
the smartphones themselves every day~\cite{manhattan}, their devices either
are not integrated into the healthcare communication system
(so that pagers are still needed), or cannot be trusted as the
system components due to lacking of the security mechanism
compliant to Health Insurance Portability and Accountability
Act (HIPAA). It is reported \cite{mdnews} that lost, stolen, and non-HIPAA
compliant healthcare mobile applications are blamed
for protected health information (PHI) breach on electronic
health record (EHR) system.

To take advantage of modern communication means supported by the mobile devices,
such as multimedia (text, image, and video) messaging and conferencing, and fully
employ the emerging technologies in healthcare industry with
HIPAA compliance, a system-wide security safeguard must
be included to defend against the potential security breaches.
Checking the HIPAA privacy rule~\cite{hipaarule}, we consider the following
three security challenges for the deployment of the mobile
devices in healthcare industry.

\begin{itemize}
\item \textit{Network security:} The user authentication and data encryption
schemes must be robust to protect PHI transmitted in the communication channel
and defend against various security attacks.
\item \textit{Dynamic security demands and fine-grained access control policies:}
Access control policies should be flexible
enough to meet the dynamic security demands in the
various scenarios that the mobile devices will be utilized.
Furthermore, the healthcare organization (e.g., hospital)
should be able to dynamically enforce system-wide access
control policies on the mobile devices to allow the
organizations to dictate their own security policies.
\item \textit{PHI propagation control and revocation:} Healthcare organizations
should be able to control the propagation of
the PHI and revoke access to the PHI that is sent to the
mobile device. For example, the healthcare organization
should be able to prevent PHI from being saved to the
device (e.g., saving a medical record as a file on the
device) or forwarded to another device (e.g., sending a
medical record to another device).
\end{itemize}

To address the aforementioned security challenges and meet
the requirements of the HIPAA regulations, we select the
Android mobile operating system as the platform to develop
our custom security framework. However, the existing security
model on the Android platform fails to address the security
challenges. In particular, Android’s current security model
supports static installation-time privileges and does not permit
the dynamic restriction of previously granted privileges.
Furthermore, Android’s security model also does not include
a mechanism to control the propagation and revocation of
sensitive data on the device.

This paper introduces a Distributed Android Security Framework
(DASF). DASF is a custom security framework for
Android-based mobile operating systems designed to provide
dynamic privilege restrictions on applications and security
policies on sensitive data on the device. Moreover, DASF
includes an application-layer networking
protocol (in regard to the OSI model) that allows the organization
(e.g., hospital, clinic, etc.) to dictate their own security
policies on the device and on data sent to the device. DASF
allows an application’s access privileges
to be flexibly set up upon invocation and adjusted on-the-fly to
meet the various security demands when dealing with sensitive
information. In mobile devices that utilize DASF, the organization
ultimately controls the security policies that the system enforces
on the user’s device and on the sensitive data that applications
receive from the server.

\begin{comment}
Therefore, the contributions that this paper provides are as
follows.  First, this paper provides a security policy
for dynamic system-wide privilege restrictions to be
enforced on applications.  Second, this paper also provides
a security policy that can be enforced on sensitive data.
The aforementioned security policies can either be utilized
by a server communicating with the mobile device or programatically
by a developer who is creating an application.
\end{comment}
Our contributions in this paper can be summarized as follows. First,
we use the medical communication system as an example to study the
security problems and the HIPAA compliance issues caused by the
mobile devices.  Second,
we design a distributed Android security framework that not only
provides dynamic system-wide security provisioning, but also protects
the privacy of the received sensitive data and therefore defeats
the PHI breach on mobile devices.  Third, we implement
a prototype Android-based medical communication system and deploy our
proposed security framework.  Our evaluation shows DASF is capable of
defeating a various security attacks while only imposing a reasonable
system overhead.


\section{Attack Model and Security Assumptions}

This work focuses on the security challenges of enforcing 
security policies on sensitive data sent to Android-based 
mobile devices and enforcing dynamic privilege restrictions. 
Moreover, this work is focused on providing the security 
guarantees at the system-level rather than at the application-level.
Thus, the security policies of the server and database 
management system are out of scope of this paper. Therefore, 
we assume that the data is protected in the server and the 
database management system by the appropriate security measures.
Furthermore, we assume that the server ensures that the 
communication channel between the server and the application 
running on the mobile device is secure (e.g., using SSL) and 
each individual involved in the communication has his/her 
security credentials (including digital certificates, public and 
private key, etc.) distributed through an out-of-band security 
channel. 

The mobile devices are not trusted. Once the mobile device 
is lost or stolen, the adversaries (who have the possession of 
the missing devices) can capture all stored data (both in flash 
memory and main memory) and may try all possible means to 
recover the sensitive data. Further, all other third party Android 
applications installed on the mobile devices (including medical 
applications) are not trusted and may launch attacks to try 
to gather sensitive information. Since the Android platform 
that we augmented with DASF is 
transparent to application packages, malicious applications 
may be installed and attempt to steal sensitive data from the 
device or the environment. We assume the communication 
parties are “semi-trusted”. That is, the user may misbehave by
themselves (e.g., trying to save or forward a patient’s MRI 
image without proper authorization) but do not conspire with 
the server. 

Therefore, we consider the following four different types of security
threats for which the system must provide protection. 

\begin{itemize}
\item \textit{Legitimate user misbehavior:} The system must prevent 
legitimate users from manually attempting to save or forward
sensitive information.
\item \textit{Malicious users:} The system must prevent sensitive information
from being recovered from the device if it is lost or stolen.
\item \textit{Application misbehavior:} The system must prevent legitimate
applications from attempting to save or forward
sensitive information that is not authorized by the organization’s
policies. For example, it is typical to write an
image to a file when it is received rather than storing
the entire image in RAM. However, writing the image to
a file may violate the organization’s policy and must be
prevented by the system.
\item \textit{Malicious applications:} The system must prevent malicious
applications from attempting to steal sensitive
information stored on the device (e.g., PHI information
authorized to be saved to the device) or from the environment
(e.g., verbal conversations occurring between
patients and doctors).
\end{itemize}

