\section{Related Work}
It is believed that 1.4 billion smartphones will be in active by the
end of 2013~\cite{rbiresearch}, and 57\% of them will run the Android
operating system. The popularity of the Android devices has made it
the top target of the mobile malware~\cite{Felt,Becher,xjiang_oak12}
and other potential attacks~\cite{Fahl,hijack}. To improve the
smartphone security defense, a number of Android system extensions
\cite{Nauman,Beresford,xjiang_11,Hornyack,Ongtang,Enck_ccs09,Lange,Andrus}
have been proposed. Apex~\cite{Nauman} proposes an Android enforcement
frame that enables flexible permission granting and resource
restrictions.  MockDroid~\cite{Beresford} is closely related to this
work in that the Android system is modified to protect the privacy.
The main difference, however, is that MockDroid protects the privacy
by adding perturbations to achieve ambiguity.  Our framework
offers a system-wide, dynamic and strict access restrictions to
protect the sensitive data.  In addition, our framework is 100\%
transparent to general Android applications.  TISSA~\cite{xjiang_11}
protects the private information leakage by an extra permission
layer. AppFence~\cite{Hornyack} provides a similar fine-grained
permission control.  Different from their schemes, our security
framework is more versatile, protecting both outgoing and 
incoming data through an on-demand dynamic security provisioning.
Saint~\cite{Ongtang} uses the design-time security policies to manage 
permissions. Kirin~\cite{Enck_ccs09} proposes a set of security rules
to mitigate malware.  L4Android~\cite{Lange} and Cells~\cite{Andrus}
provide the improved OS isolation for security.  

%TaintDroid~\cite{taintdroid} provides a information flow monitoring
%system to detect the potential privacy leakage. This idea is adopted
%in our system framework to enforce the data access privilege and
%propogation restrictions.

TaintDroid~\cite{taintdroid} provides a information flow monitoring
system to detect the potential privacy leakage. This idea is adopted
in our system framework to enforce the data access privilege control
and propagation restrictions.
Static analysis has been used to detect privacy leak in Android
applications~\cite{Batyuk} and potential security
vulnerabilities~\cite{Lu_ccs12}.  Crowdroid~\cite{Burguera} analyzes
the system calls dynamically to detect the malware.  Android
permission specification has been thoroughly studied in
Stowaway~\cite{Felt_ccs11} and PScout~\cite{Au_ccs12}.  As discussed
previously, the permission based Android security system is
insufficient to address the security challenges in the application
compliant with the HIPAA rule.

