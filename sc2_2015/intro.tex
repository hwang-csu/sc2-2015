\section{Introduction}

The incorporation of mobile devices in the healthcare industry can prove to be
another beneficial tool for medical professionals, as doctors and nurses
frequently communicate with each other to provide a diagnosis and treatment.
While the appeal of modern communication through mobile devices, e.g.,
smartphones or tablets, has reached almost every other industry, the following
observations witness the status of mobile devices in healthcare industry.
First, a significant number of physicians and nurses are still relying on
traditional clinical communication equipment, including pagers, phones and fax
machine, yet a last generation wireless communication technology. Second, even
though as many as 81\% healthcare professionals own and use the smartphones
themselves every day~\cite{manhattan}, their devices either are not integrated
into the healthcare communication system (so that pagers are still needed), or
cannot be trusted as the system components due to lacking of the security
mechanism compliant to Health Insurance Portability and Accountability Act
(HIPAA). It is reported \cite{mdnews} that lost, stolen, and non-HIPAA
compliant healthcare mobile applications are blamed for protected health
information (PHI) breach on electronic health record (EHR) systems. 

To take advantage of modern communication means supported by the mobile
devices, such as multimedia (text, image, and video) messaging and
conferencing, and fully employ the emerging technologies in healthcare industry
with HIPAA compliance, a system-wide security safeguard must be included to
defend against the potential security breaches. Checking the HIPAA privacy
rule~\cite{hipaarule}, we consider the following three security challenges for
the deployment of the mobile devices in healthcare industry.

\begin{itemize}
\item \textit{Network security:} The user authentication and data
  encryption schemes must be robust to protect PHI transmitted in the
  communication channel and defend against various security attacks.
\item \textit{Sensitive data flow confinement:} The service
  administrator should be able to control the propagation of the
  sensitive data and, if necessary, revoke the access to protected
  data that is sent to the mobile devices. For example, the healthcare
  organization should prevent PHI from being saved to the devices
  (e.g., saving a medical record as a file) or being forwarded to
  another device.
\item \textit{Dynamic security demands and fine-grained access control
  policies:} Access control policies should be flexible enough to meet
  the dynamic security demands in the various scenarios that the
  mobile devices are utilized. For example, the service administrator
  may want to dynamically enforce system-wide access control policies
  on the mobile devices to allow the organizations to dictate their
  own security policies.  
%\item \textit{PHI propagation control and revocation:} Healthcare
%  organizations should be able to control the propagation of the PHI
%  and revoke access to the PHI that is sent to the mobile device. For
%  example, the healthcare organization should be able to prevent PHI
%  from being saved to the device (e.g., saving a medical record as a
%  file on the device) or being forwarded to another device (e.g., sending a
%  medical record to another device). 
\end{itemize}
\begin{comment}
While the network security can be guarded by existing security
protocols, e.g., TLS/SSL, there is a non-trivial technical hurdle to
address the latter two challenges to protect the data privacy.
This work focuses on dynamic security provisioning and sensitive data
propagation control.   
%In this work, we select the Android
%mobile operating system as the platform to develop our custom security
%platform. 
%To address the aforementioned security challenges and meet the
%requirements of the HIPAA regulations, we select the Android mobile
%operating system as the platform to develop our custom security
%framework. However, the existing security model on the Android
%platform fails to address the security challenges. In particular,
The current mobile operating systems, e.g., Android, only support
%Android's current security model supports 
static installation-time privilege control and do not support the
dynamic restriction of previously granted privileges. Furthermore,
their security model also does not include a mechanism to control the
propagation and revocation of sensitive data on the device. One may
note that the above data protection can be addressed in the
application level by carefully designing and developing a
privacy-preserving application. Unfortunately, the application
level protection can be easily compromised due to the nature of the
mobile devices and the limitations of the mobile operating
systems. First, a stolen or lost device would give the adversary a
total access of its stored data (e.g., by sideloading a malicious
application that performs a full memory dump).  Second, a misbehaving
user may also launch the above attack or simply acquires the system
privilege (e.g., by rooting the device) to gain free access to the
data stored in the memory. Therefore, we argue that a standalone
mobile operating system is lacking the security
measures that can defend against various security attacks toward the
sensitive data. In this work, we propose a novel security design
concept for mobile operating system: a distributed security
framework (DSF).  DSF is based on a centralized (e.g., a cloud or a
server) security system, and pushes the security boundary along the
way to the mobile devices that include the mobile operating system
(e.g., Android). In our design, user devices are tightly 
coupled with the system server so that DSF makes sure all system
activities on the device side are compliant with the system security
policies. The loss of coupling between the device and the server,
including stolen, lost and system rooting, is considered as the
security attack that results in the security countermeasures, e.g.,
remote wipe-out. 
\end{comment}
This paper uses the healthcare communication application as a platform to
demonstrate the design and development of a Distributed Android Security
Framework (DASF), a customized security framework for Android-based mobile
operating systems designed to provide dynamic privilege restrictions on
applications and security policies on sensitive data on the device. The
coupling between the Android and a server is achieved by using secure
heart-beat messages through a system security channel established during
installation time.  In addition, DASF includes an application-layer networking
protocol (in regard to the OSI model) that allows the organization (e.g.,
hospital, clinic, etc.) to dictate their own security policies on the device
and on data sent to the device. DASF allows an application's access privileges
to be flexibly set up upon invocation and adjusted on-the-fly to meet the
various security needs when dealing with sensitive information. In mobile
devices that install DASF, the organization ultimately controls the security
policies that the system enforces on the user's device and on the sensitive
data that applications receive.

\begin{comment}
Therefore, the contributions that this paper provides are as
follows.  First, this paper provides a security policy
for dynamic system-wide privilege restrictions to be
enforced on applications.  Second, this paper also provides
a security policy that can be enforced on sensitive data.
The aforementioned security policies can either be utilized
by a server communicating with the mobile device or 
by a developer who is creating an application.
\end{comment}

%Our contributions in this paper can be summarized as follows. First,
%we use the medical communication system as an example to study the
%security problems and the HIPAA compliance issues caused by the
%mobile devices.  Second,
%we design a distributed Android security framework that not only
%provides dynamic system-wide security provisioning, but also protects
%the privacy of the received sensitive data and therefore defeats
%the PHI breach on mobile devices.  Third, we implement
%a prototype Android-based medical communication system and deploy our
%proposed security framework.  Our evaluation shows DASF is capable of
%defeating a variety of security attacks while only imposing a reasonable
%system overhead.

This paper makes the following contributions:
\begin{itemize}
	\item We study the security problems and HIPAA compliance issues
	caused by mobile devices, using a medical communication system as an example.
	\item We design a distributed Android security framework that not only
	provides dynamic system-wide security provisioning, but also protects
	the privacy of the received sensitive data, and therefore protects against PHI
	breaches on mobile devices.
	\item We implement a prototype Android-based medical communication
	system and deploy our proposed security framework.  Our evaluation shows DASF
	is capable of defeating a variety of security attacks while imposing a
	reasonable system overhead.
\end{itemize}

The remainder of this paper proceeds as follows. Section II discusses our
threat model and security assumptions. Section III formalizes the model of our
security framework. Section IV discusses our prototype's implementation.
Section V evaluates the security and performance of our prototype.  Section VI
discusses relevant related works. Finally, Section VII concludes.


\section{Threat Model and Security Assumptions}
\begin{comment}
This work focuses on the security challenges of enforcing security
policies on sensitive data sent to Android-based mobile devices and
enforcing dynamic privilege restrictions.
%Moreover, this work is focused on providing the security 
%guarantees at the system-level rather than at the application-level.
Thus, the security policies of the server and database management
system are out of scope of this study. And we assume that the data is
protected in the server (or cloud) and the database management system
by the appropriate security measures. Furthermore, we assume that the
server ensures that the communication channel between the server and
the application running on the mobile device is secure (e.g., SSL) and
each individual involved in the communication has his/her security
credentials (including digital certificates, public and private key,
etc.) distributed through an out-of-band security channel.  
\end{comment}
We assume that the Android OS, which includes the Linux kernel and Android's
system services (e.g., Activity Manager Service, Package Manager Service), is
trusted.
%We assume that an adversary may have physical access to the
%device, and can capture stored data and try to recover the sensitive data by
%all possible means.
We assume that all third-party Android applications installed on the mobile
devices (including \textit{medical applications}) are untrusted and may launch
attacks to try to gather sensitive information. However, we assume that these
third-party applications do not have root privileges, and do not collude with
other apps (e.g., covert channels).  Since the Android platform that we
augmented with DASF is transparent to application packages, malicious
applications may be installed and attempt to steal sensitive data from the
device or the environment. We assume the communication parties are
“semi-trusted”. That is, the user may misbehave by themselves (e.g., trying to
save or forward a patient's MRI image without proper authorization), but does
not conspire with the server. 

%Mobile devices are not trusted. Once the mobile device is lost or
%stolen, the adversaries can capture stored data and may try all
%possible means to recover the sensitive data.
%Further, all other third
%party Android applications installed on the mobile devices (including
%\textit{medical applications}) are not trusted and may launch attacks
%to try to gather sensitive information. Since the Android platform that we
%augmented with DASF is transparent to application packages, malicious
%applications may be installed and attempt to steal sensitive data from the
%device or the environment. We assume the communication parties are
%“semi-trusted”. That is, the user may misbehave by themselves (e.g., trying to
%save or forward a patient's MRI image without proper authorization) but do not
%conspire with the server. 

Therefore, we consider the following four different types of security
threats for which the system must provide protection. 

\begin{itemize}
\item \textit{Legitimate user misbehavior:} The system must prevent legitimate
users from manually attempting to save or forward sensitive information.
\item \textit{Malicious users:} The system must prevent sensitive information
from being recovered from the device if it is lost or stolen.
\item \textit{Application misbehavior:} The system must prevent legitimate
applications from attempting to save or forward sensitive information that is
not authorized by the organization's policies. For example, it is typical to
write an image to a file when it is received rather than storing the entire
image in RAM. However, writing the image to a file may violate the
organization's policy and must be prevented by the system.
\item \textit{Malicious applications:} The system must prevent malicious
applications from attempting to steal sensitive information stored on the
device (e.g., PHI information authorized to be saved to the device) or from the
environment (e.g., verbal conversations occurring between patients and
doctors).
\end{itemize}

