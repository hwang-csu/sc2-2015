%\vspace{-3mm}
\section{Conclusion and Future Work}

This paper describes a distributed security framework (DASF) for
Android-based mobile operating systems designed to provide dynamic
privilege restrictions on applications and security policies on
sensitive data sent to the device.  Unlike Android's current security
model, DASF allows an application's permissions to be restricted
dynamically to address the security concerns of mobile devices being
used in sensitive environments.  Further, DASF also allows the
organization to remain in control of the sensitive data that it sends
to the mobile device by imposing security policies on that data.  We
implement a prototype of DASF by creating a system service to enforce
privilege restrictions on applications, and our experiments show that
DASF addresses the limitations of Android's current security model
while imposing a reasonable performance overhead.  Our future work
includes the system enhancement in the following aspects.  First, we
need to develop a mechanism to differentiate the network brokeage and
security attacks.  Ideally, our DASF will allow a short disconnect
operation without sacrificing the security countermeasure
capabilities. Second, we will start our development of DASF on a
native Android system instead of the TaintDroid to further improve the
system latency performance and enhance the security performance as
discussed previously.  



%a custom \textit{InputStream} to 
%handle our message protocol, and by
%placing hooks in the Android system to enforce
%the security policies.  Further, we built a server
%module to send data and security policies to
%the device to demonstrate that the security
%policies can be dynamically imposed on both
%an application's privileges and on sensitive
%data.  Our experiments shows that DASF
%addresses the limitations of
%Android's current security model while
%imposing an reasonable performance overhead.
